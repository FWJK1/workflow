\documentclass{beamer}
\usetheme{CambridgeUS}
\usecolortheme{dolphin}
\setbeamertemplate{navigation symbols}{}


\usepackage{helvet}
\usepackage{graphicx}
\usepackage{svg}
\usepackage{hyperref}
\usepackage[backend=bibtex,style=numeric]{biblatex}
\usepackage{graphicx}
\usepackage{xcolor}



\addbibresource{/home/fitsl/symlinks/full_zotero_library.bib}
\title{Complex Systems Workflow}
\date{December 10th, 2025}
\author{X}

\begin{document}
\frame{\titlepage}


\begin{frame}{Outline}
    \tableofcontents
\end{frame}


\section{Philosophy: Modular, Replicable, Automated}

\begin{frame}{The Whole Point}
    \begin{columns}[t]
        \begin{column}[t]{.3\textwidth}
            \textbf{Learn}
            \begin{itemize}
                \item Memorize
                \item Categorize
                \item Synthesize
                \item Connect
                \item Expand
            \end{itemize}
        \end{column}
        \begin{column}[t]{.3\textwidth}
            \textbf{Create}
            \begin{itemize}
                \item Literature
                \item Explore
                \item Hypothesize
                \item Test
                \item Replicate
                \item Analyze
            \end{itemize}
        \end{column}
        \begin{column}[t]{.3\textwidth}
            \textbf{Communicate}
            \begin{itemize}
                \item Visualize
                \item Stabilize
                \item Consolidate
                \item Format
                \item Share
            \end{itemize}
        \end{column}
    \end{columns}

    \vspace{.5 cm}
    \centering
    Find $\to$ Read $\to$ \textcolor{teal}{Notes} $\to$ \textcolor{teal}{Ideas} $\to$ \textcolor{teal}{Experiment Design} $\to$ \textcolor{blue}{Code} $\to$  \textcolor{blue}{Data} $\to$  \textcolor{blue}{Figures} $\to$  \textcolor{blue}{Writing} $\to$ Publishing $\to$ Archiving
\end{frame}

\begin{frame}{How to Get There}
    \centering
    \textbf{Complex Systems = Data Science + Physics + Epistemology}
    \vspace{.5cm}
    \begin{itemize}
        \item \textbf{Modular}:  Playful, expandable, iterable
        \item \textbf{Replicable}: Collaboration without subscription, fixing, expansion
        \item \textbf{Automated}: Systems require computation
    \end{itemize}
    \vspace{.5cm}
    \centering
    \textbf{Consolidated Pipeline of Open Source Local Software}
\end{frame}

\begin{frame}{Moral and Virtue Arguments}
    \begin{itemize}
        \item \textbf{Moral Thought Experiment}: AI-assisted automation VS automation with AI. Consider environmental, social, personal ramifications. \cite{salamancaMicrosoftGoogleSay2025}
        \item \textbf{Virtue}: Better to understand the things you are using well.
        \item \textbf{Moral}: Better to live in world where skill and understanding are valuable and personal rather than auctioned off for false convenience.
        \item \textbf{Both}: How you do what you do should be a decision you make, not one that corporations make for you. \cite{HowGoogleTook}
    \end{itemize}
\end{frame}

\begin{frame}{Technical and Self-Interested Arguments}
    \begin{columns}
        \begin{column}{.48\textwidth}
            \begin{itemize}
                \item \textbf{Venal Self Interest}: Pipelining as a skillset (the age of the ipynb is fading)
                \item \textbf{Version Control}
                \item \textbf{Regular Control}
                \item \textbf{Ongoing Access}
                \item \textbf{\textit{Clearer opportunities for improvement}}
                \item \textbf{Fun!}
            \end{itemize}
        \end{column}
        \begin{column}{.48\textwidth}
            \includegraphics[width=\textwidth]{images/Elves_over.png}
        \end{column}
    \end{columns}
\end{frame}

\section{What I Do: Learning Tools}

\begin{frame}{Class Notes}
    \begin{itemize}
        \item In Class: pen and paper
        \item Later: transferred, not transcribed, to Obsidian
        \item Clarified with wikipedia, textbooks, youtube, AI
        \item Categorize, Connect, Store, Label
        \item Importance of not being a freak about it.
    \end{itemize}
\end{frame}

\begin{frame}{Research Notes}
    \begin{itemize}
        \item Zotero first
        \item Annotate and highlight digitally or on paper (if hard to focus on)
        \item Larger ideas and concepts in citation page in zotero
        \item Ideas integrated into conceptual  'evergreen' notes
    \end{itemize}
\end{frame}

\begin{frame}{Flash Cards}
    \begin{itemize}
        \item Obsidian $\to$ ANKI
        \item Flashcards of formulas, phrases, ideas that I think will anchor my memory for things that are important to me
        \item Importance of not being a freak about it, seriously, we mean it.
    \end{itemize}
\end{frame}

\begin{frame}{Live Demo}
    Live Demo
\end{frame}

\section{What I Do: Coding, Paths, Figures, Latex}
\begin{frame}{Create}
    \centering
    \textbf{The Pipeline}: experiment ideas in Obsidian turn into code in \texttt{src} uses data in \texttt{data} to make figures in \texttt{latex/figures} which \texttt{latex/project.tex}, possibly drafted in Obsidian, complies to share results.
\end{frame}


\begin{frame}{Computational Experiments}
    A whole slide deck in itself....
\end{frame}


\begin{frame}{Coding Practices}
    \begin{itemize}
        \item \textbf{Mise en Place}: Clean Cutting Board, Clean Mind. Build a recipe, then prepare ingredients, then put together.
        \item \textbf{DRY}: Be kind to your future self! Like getting a glass of water after going out drinking.
        \item \textbf{Modular}: Organized functionality = iterable functionality.
        \item \textbf{Paths}: Never raw strings, (python: Pathlib) and as limited absolute paths as possible
        \item \textbf{Reproducibility}: if in python, use conda/uv or at least a regularly pip list.
        \item Make stuff so that you can use it in later projects!
        \item Tests, Types, \textbf{Dataclasses}
        \item \textbf{Doing everything like you want to be better at doing it next time.}
    \end{itemize}
\end{frame}

\begin{frame}{Data Management}
    \begin{itemize}
        \item Don't put it on github
        \item Consider scale before you start
        \item Standardize, Log, and \textit{save a record of all transformations}, ideally in a replicable script
    \end{itemize}
\end{frame}

\begin{frame}{Latex}
    \begin{itemize}
        \item Local is better then overleaf: version control, directly piping in figures and results from code, potential for more automated writing.
        \item Write your own snippets and macros
        \item Latex extensions in vs code make it really easy
        \item Exporting full zotero bib allows for really quick citations.
        \item Live Demo!
    \end{itemize}
\end{frame}


\begin{figure}[h]
    \centering
    \includegraphics[width=.8\textwidth]{figures/gender_bar.png}
    \caption{Gender distribution, which we will here claim has something to do with $R^2$ \cite{CoefficientDetermination2025}}
    \label{fig:gender_bar}
\end{figure}

\begin{frame}{Version Control: GIT}
    \begin{itemize}
        \item Commit when you might break something; push when you've changed or fixed something; PR when you've finished a defined subproject
        \item Easy to link an employer a cool github.
        \item Easy to bring in collaborators.
        \item Test: guess the GOODNESS levels of these three options:
              \begin{itemize}
                  \item VS Code extension
                  \item Terminal
                  \item Github GUI
              \end{itemize}
    \end{itemize}
\end{frame}

\section{}

\begin{frame}[allowframebreaks]{References}
    \printbibliography
\end{frame}

\end{document}